\documentclass[12pt,a4paper]{article}
\usepackage[UTF8]{ctex}
\usepackage{amsmath,amssymb,amsthm}
\usepackage{graphicx}
\usepackage{geometry}
\usepackage{hyperref}
\usepackage{algorithm}
\usepackage{algorithmic}
\usepackage{listings}
\usepackage{xcolor}

\geometry{margin=2.5cm}

\title{第10章\quad 运动规划}
\author{现代机器人学:力学、规划与控制\\
Kevin M. Lynch 和 Frank C. Park\\
中文翻译}
\date{}

\begin{document}

\maketitle

\section{引言}

运动规划是在满足其他约束(如关节限位或力矩限制)的情况下,找到从起始状态到目标状态且避开环境中障碍物的机器人运动的问题。运动规划是机器人学中最活跃的子领域之一,也是整本书的主题。本章的目的是提供一些常用技术的实用概述,主要以机械臂和移动机器人作为示例系统(图10.1)。

本章首先简要概述运动规划,然后介绍基础材料,包括配置空间障碍物和图搜索。最后总结几种不同的规划方法。

\section{运动规划概述}

运动规划中的一个关键概念是配置空间(configuration space),简称C空间。C空间$C$中的每个点对应于机器人的唯一配置$q$,机器人的每个配置都可以表示为C空间中的一个点。例如,具有$n$个关节的机械臂的配置可以表示为$n$个关节位置的列表,$q = (\theta_1, \ldots, \theta_n)$。自由C空间$C_{\text{free}}$由机器人既不穿透障碍物也不违反关节限位的配置组成。

在本章中,除非另有说明,我们假设$q$是一个$n$维向量,且$C \subset \mathbb{R}^n$。通过一些推广,本章的概念也适用于非欧几里得C空间,如$C = SE(3)$。

可用于驱动机器人的控制输入写为$m$维向量$u \in U \subset \mathbb{R}^m$,其中对于典型的机械臂$m = n$。如果机器人具有二阶动力学(如机械臂),且控制输入是力(等价地,加速度),则机器人的状态由其配置和速度定义,$x = (q, v) \in X$。对于$q \in \mathbb{R}^n$,通常我们写$v = \dot{q}$。如果我们可以将控制输入视为速度,则状态$x$就是配置$q$。记号$q(x)$表示对应于状态$x$的配置$q$,且$X_{\text{free}} = \{x | q(x) \in C_{\text{free}}\}$。

机器人的运动方程写为
\begin{equation}
\dot{x} = f(x, u)
\end{equation}

或者,以积分形式表示:
\begin{equation}
x(T) = x(0) + \int_0^T f(x(t), u(t))dt
\end{equation}

\subsection{运动规划问题的类型}

根据上述定义,运动规划问题的相当广泛的规范如下:

给定初始状态$x(0) = x_{\text{start}}$和期望的最终状态$x_{\text{goal}}$,找到时间$T$和控制集$u: [0, T] \to U$,使得运动(10.2)满足$x(T) = x_{\text{goal}}$且对所有$t \in [0, T]$有$q(x(t)) \in C_{\text{free}}$。

假设有反馈控制器(第11章)可用于确保紧密跟随规划的运动$x(t), t \in [0, T]$。还假设有机器人和环境的精确几何模型可用于在运动规划期间评估$C_{\text{free}}$。

基本问题有许多变体;下面讨论一些。

\textbf{路径规划与运动规划。}路径规划问题是通用运动规划问题的子问题。路径规划是纯粹几何问题,即找到从起始配置$q(0) = q_{\text{start}}$到目标配置$q(1) = q_{\text{goal}}$的无碰撞路径$q(s), s \in [0, 1]$,而不关心动力学、运动持续时间或对运动或控制输入的约束。假设路径规划器返回的路径可以时间标定以创建可行轨迹(第9章)。这个问题有时被称为钢琴搬运工问题,强调对杂乱空间几何的关注。

\textbf{控制输入:$m = n$与$m < n$。}如果控制输入数$m$少于自由度$n$,则机器人无法跟随许多路径,即使它们是无碰撞的。例如,汽车有$n = 3$(底盘在平面中的位置和方向),但$m = 2$(前后运动和转向);它不能直接侧向滑入停车位。

\textbf{在线与离线。}需要立即结果的运动规划问题(可能因为障碍物出现、消失或不可预测地移动)需要快速、在线的规划器。如果环境是静态的,则较慢的离线规划器可能就足够了。

\textbf{最优与满意。}除了到达目标状态,我们可能希望运动规划最小化(或近似最小化)成本$J$,例如:
\begin{equation}
J = \int_0^T L(x(t), u(t))dt
\end{equation}

例如,用$L = 1$最小化得到时间最优运动,而用$L = u^T(t)u(t)$最小化得到“最小努力”运动。

\textbf{精确与近似。}我们可能满足于最终状态$x(T)$足够接近$x_{\text{goal}}$,例如,$\|x(T) - x_{\text{goal}}\| < \epsilon$。

\textbf{有或无障碍物。}即使在没有障碍物的情况下,运动规划问题也可能具有挑战性,特别是如果$m < n$或需要最优性。

\subsection{运动规划器的性质}

规划器必须符合上述运动规划问题的性质。此外,规划器可以通过以下性质来区分。

\textbf{多查询与单查询规划。}如果要求机器人在不变的环境中解决多个运动规划问题,可能值得花费时间构建准确表示$C_{\text{free}}$的数据结构。然后可以搜索该数据结构以高效解决多个规划查询。单查询规划器从头开始解决每个新问题。

\textbf{“随时”规划。}随时规划器是在找到第一个解后继续寻找更好解的一种规划器。规划器可以在任何时候停止,例如当指定的时间限制已过时,并返回最佳解。

\textbf{完备性。}如果保证在有限时间内找到解(如果存在),并在没有可行运动规划时报告失败,则称运动规划器是完备的。较弱的概念是分辨率完备性。如果保证在问题的离散化表示的分辨率下找到解(如果存在),例如$C_{\text{free}}$的网格表示的分辨率,则规划器是分辨率完备的。最后,如果找到解的概率(如果存在)随着规划时间趋于无穷而趋于1,则规划器是概率完备的。

\textbf{计算复杂度。}计算复杂度是指规划器运行所需时间或所需内存量的特征。这些是根据规划问题的描述来测量的,例如C空间的维度或机器人和障碍物表示中的顶点数。例如,规划器运行的时间可能是$n$(C空间的维度)的指数。计算复杂度可以用平均情况或最坏情况来表示。一些规划算法容易进行计算复杂度分析,而其他则不容易。

\subsection{运动规划方法}

没有适用于所有运动规划问题的单一规划器。下面是对许多可用运动规划器的广泛概述。细节留给指示的章节。

\textbf{完备方法(第10.3节)。}这些方法专注于$C_{\text{free}}$的几何或拓扑的精确表示,确保完备性。对于除简单或低自由度问题外的所有问题,这些表示在数学上或计算上难以推导。

\textbf{网格方法(第10.4节)。}这些方法将$C_{\text{free}}$离散化为网格,并在网格中搜索从$q_{\text{start}}$到目标区域中网格点的运动。该方法的修改可以离散化状态空间或控制空间,或者它们可以使用多尺度网格来细化$C_{\text{free}}$在障碍物附近的表示。这些方法相对容易实现,可以返回最优解,但对于固定分辨率,存储网格所需的内存和搜索时间随空间维数呈指数增长。这限制了该方法仅适用于低维问题。

\textbf{采样方法(第10.5节)。}通用采样方法依赖于随机或确定性函数从C空间或状态空间中选择样本;评估样本是否在$X_{\text{free}}$中的函数;确定“最近”的先前自由空间样本的函数;以及尝试从先前样本连接到或移向新样本的局部规划器。此过程构建表示机器人可行运动的图或树。采样方法易于实现,往往是概率完备的,甚至可以解决高自由度运动规划问题。解往往是满意的,不是最优的,并且很难表征计算复杂度。

\textbf{虚拟势场(第10.6节)。}虚拟势场在机器人上产生将其拉向目标并推离障碍物的力。该方法相对容易实现,即使对于高自由度系统也是如此,并且评估速度快,通常允许在线实现。缺点是势函数中的局部最小值:机器人可能卡在吸引力和排斥力抵消但机器人不在目标状态的配置中。

\textbf{非线性优化(第10.7节)。}通过用有限数量的设计参数表示路径或控制,可以将运动规划问题转换为非线性优化问题,例如多项式或傅里叶级数的系数。问题是求解最小化成本函数同时满足对控制、障碍物和目标的约束的设计参数。虽然这些方法可以产生接近最优的解,但它们需要解的初始猜测。因为目标函数和可行解空间通常不是凸的,优化过程可能卡在远离可行解的地方,更不用说最优解了。

\textbf{平滑(第10.8节)。}规划器找到的运动通常是急动的。可以在运动规划器的结果上运行平滑算法以改善平滑度。

近年来,主要趋势是采样方法,它们易于实现并且可以处理高维问题。

\section{基础}

在讨论运动规划算法之前,我们建立许多算法中使用的概念:配置空间障碍物、碰撞检测、图和图搜索。

\subsection{配置空间障碍物}

确定处于配置$q$的机器人是否与已知环境碰撞通常需要涉及环境的CAD模型和机器人的复杂操作。有许多免费和商业软件包可以执行此操作,我们在此不深入讨论。对于我们的目的,知道工作空间障碍物将配置空间$C$划分为两个集合就足够了:自由空间$C_{\text{free}}$和障碍空间$C_{\text{obs}}$,其中$C = C_{\text{free}} \cup C_{\text{obs}}$。关节限位被视为配置空间中的障碍物。

有了$C_{\text{free}}$和$C_{\text{obs}}$的概念,路径规划问题简化为在障碍物$C_{\text{obs}}$中为点机器人找到路径的问题。如果障碍物将$C_{\text{free}}$分成不同的连通分量,且$q_{\text{start}}$和$q_{\text{goal}}$不在同一连通分量中,则没有无碰撞路径。

C障碍物的显式数学表示可能极其复杂,因此很少精确表示C障碍物。尽管如此,C障碍物的概念对于理解运动规划算法非常重要。这些想法最好通过例子来说明。

\subsubsection{2R平面机械臂}

图10.2显示了2R平面机器人臂,配置为$q = (\theta_1, \theta_2)$,在工作空间中的障碍物A、B和C之间。机器人的C空间由平面的一部分表示,其中$0 \leq \theta_1 < 2\pi, 0 \leq \theta_2 < 2\pi$。然而,记住第2章,C空间的拓扑是环面(或甜甜圈),因为正方形在$\theta_1 = 2\pi$的边连接到边$\theta_1 = 0$;类似地,$\theta_2 = 2\pi$连接到$\theta_2 = 0$。$\mathbb{R}^2$的正方形区域是通过在$\theta_1 = 0$和$\theta_2 = 0$处两次切割甜甜圈表面,并将其平铺在平面上获得的。

图10.2右侧的C空间显示了表示为C障碍物的工作空间障碍物A、B和C。位于C障碍物内的任何配置对应于工作空间中机器人臂对障碍物的穿透。显示了从一种配置到另一种配置的机器人臂的自由路径,既在工作空间也在C空间中。路径和障碍物说明了C空间的拓扑。注意障碍物将$C_{\text{free}}$分成三个连通分量。

\subsubsection{圆形平面移动机器人}

图10.3显示了圆形移动机器人的俯视图,其配置由其中心的位置$(x, y) \in \mathbb{R}^2$给出。机器人在具有单个障碍物的平面中平移(不旋转地移动)。相应的C障碍物通过将工作空间障碍物“增长”(扩大)移动机器人的半径来获得。该C障碍物外的任何点表示机器人的自由配置。图10.4显示了两个障碍物的工作空间和C空间,表明在这种情况下移动机器人无法在两个障碍物之间通过。

\subsubsection{可平移的多边形平面移动机器人}

图10.5显示了在存在多边形障碍物的情况下平移的多边形移动机器人的C障碍物。C障碍物通过将机器人沿障碍物边界滑动并追踪机器人参考点的位置来获得。

\subsubsection{可平移和旋转的多边形平面移动机器人}

图10.6说明了如果现在允许机器人旋转,图10.5的工作空间障碍物和三角形移动机器人的C障碍物。C空间现在是三维的,由$(x, y, \theta) \in \mathbb{R}^2 \times S^1$给出。三维C障碍物是在角度$\theta \in [0, 2\pi)$处的二维C障碍物切片的并集。即使对于这个相对低维的C空间,C障碍物的精确表示也相当复杂。因此,很少精确描述C障碍物。

\subsection{到障碍物的距离和碰撞检测}

给定C障碍物$B$和配置$q$,设$d(q, B)$是机器人和障碍物之间的距离,其中
\begin{align}
d(q, B) &> 0 \quad \text{(不与障碍物接触)} \\
d(q, B) &= 0 \quad \text{(接触)} \\
d(q, B) &< 0 \quad \text{(穿透)}
\end{align}

距离可以定义为机器人和障碍物分别最接近的两点之间的欧几里得距离。

距离测量算法是确定$d(q, B)$的算法。碰撞检测例程确定对于任何C障碍物$B_i$是否有$d(q, B_i) \leq 0$。碰撞检测例程返回二进制结果,并且可能或可能不在其核心使用距离测量算法。

一种流行的距离测量算法是Gilbert–Johnson–Keerthi (GJK)算法,它高效计算两个凸体之间的距离,可能由三角网格表示。任何机器人或障碍物都可以被视为多个凸体的并集。该算法的扩展用于机器人学、图形和游戏物理引擎中的许多距离测量算法和碰撞检测例程。

更简单的方法是将机器人和障碍物近似为重叠球体的并集。近似必须始终是保守的——近似必须覆盖对象的所有点——这样如果碰撞检测例程指示自由配置$q$,则我们保证实际几何是无碰撞的。随着机器人和障碍物表示中球体数量的增加,近似更接近实际几何。图10.7显示了一个例子。

给定在$q$处的机器人,由$k$个半径为$R_i$、中心在$r_i(q)$的球体表示,$i = 1, \ldots, k$,以及由$\ell$个半径为$B_j$、中心在$b_j$的球体表示的障碍物$B$,$j = 1, \ldots, \ell$,机器人和障碍物之间的距离可以计算为
\begin{equation}
d(q, B) = \min_{i,j} \|r_i(q) - b_j\| - R_i - B_j
\end{equation}

除了确定机器人的特定配置是否碰撞外,另一个有用的操作是确定机器人在特定运动段期间是否碰撞。虽然已经为特定对象几何和运动类型开发了精确解,但通用方法是在精细间隔的点处对路径进行采样,并“增长”机器人以确保如果两个连续配置对于增长的机器人是无碰撞的,则实际机器人在两个配置之间扫过的体积也是无碰撞的。

\subsection{图和树}

许多运动规划器显式或隐式地将C空间或状态空间表示为图。图由节点集合$N$和边集合$E$组成,其中每条边$e$连接两个节点。在运动规划中,节点通常表示配置或状态,而节点$n_1$和$n_2$之间的边表示能够从$n_1$移动到$n_2$而不穿透障碍物或违反其他约束。

图可以是有向的或无向的。在无向图中,每条边是双向的:如果机器人可以从$n_1$行驶到$n_2$,则它也可以从$n_2$行驶到$n_1$。在有向图(简称digraph)中,每条边只允许在一个方向上行驶。相同的两个节点之间可以有两个边,允许在相反方向上行驶。

图也可以是加权的或未加权的。在加权图中,每条边都有与遍历它相关的正成本。在未加权图中,每条边具有相同的成本(例如,1)。因此,我们考虑的最通用类型的图是加权有向图。

树是没有循环的有向图,其中(1)没有循环,且(2)每个节点最多有一个父节点(即,最多有一条指向该节点的边)。树有一个没有父节点的根节点和多个没有子节点的叶节点。图10.8说明了有向图、无向图和树。

给定$N$个节点,任何图都可以由矩阵$A \in \mathbb{R}^{N \times N}$表示,其中矩阵的元素$a_{ij}$表示从节点$i$到节点$j$的边的成本;零或负值表示节点之间没有边。图和树可以更紧凑地表示为节点列表,每个节点都有指向其邻居的链接。

\subsection{图搜索}

一旦自由空间表示为图,可以通过在图中搜索从起点到目标的路径来找到运动规划。最强大和最流行的图搜索算法之一是$A^*$(读作“A星”)搜索。

\subsubsection{$A^*$搜索}

$A^*$搜索算法在路径成本简单地是沿路径的正边成本之和时,高效找到图上的最小成本路径。给定由节点集合$N = \{1, \ldots, N\}$描述的图,其中节点1是起始节点,以及边集合$E$,$A^*$算法使用以下数据结构:

\begin{itemize}
\item 排序列表OPEN,包含仍需探索的节点,以及列表CLOSED,包含已经探索过的节点;
\item 矩阵cost[node1,node2]编码边集合,其中正值对应于从node1移动到node2的成本(负值表示不存在边);
\item 数组past\_cost[node],表示到目前为止从起始节点到达节点node的最小成本;以及
\item 由数组parent[node]定义的搜索树,其中包含每个节点,指向到目前为止从起始节点到该节点的最短路径中位于其前面的节点。
\end{itemize}

为了初始化搜索,构建矩阵cost以编码边,列表OPEN初始化为起始节点1,到达起始节点的成本(past\_cost[1])初始化为0,对于node $\in \{2, \ldots, N\}$的past\_cost[node]初始化为无穷大(或大数),表示目前我们不知道到达这些节点的成本。

在算法的每一步,OPEN中的第一个节点从OPEN中移除并称为current。节点current也被添加到CLOSED。OPEN中的第一个节点是使通过该节点的最佳路径到目标的估计总成本最小的节点。估计成本计算为
\begin{equation}
\text{est\_total\_cost}[node] = \text{past\_cost}[node] + \text{heuristic\_cost\_to\_go}(node)
\end{equation}

其中heuristic\_cost\_to\_go(node) $\geq 0$是从节点到目标的实际成本到目标的乐观(低估)估计。对于许多路径规划问题,启发式的适当选择是到目标的直线距离,忽略任何障碍物。

因为OPEN是根据估计总成本排序的列表,在OPEN中的正确位置插入新节点需要小的计算代价。

如果节点current在目标集中,则搜索完成,路径从父链接重构。如果不是,对于图中current的每个邻居nbr(也不在CLOSED中),计算nbr的tentative\_past\_cost为past\_cost[current] + cost[current,nbr]。如果tentative\_past\_cost < past\_cost[nbr],则nbr可以以比先前已知更低的成本到达,因此past\_cost[nbr]设置为tentative\_past\_cost,parent[nbr]设置为current。然后根据其估计总成本将节点nbr添加(或移动)到OPEN中。

算法然后返回到主循环的开始,从OPEN中移除第一个节点并称其为current。如果OPEN为空,则没有解。

$A^*$算法保证返回最小成本路径,因为只有在它们具有所有节点的最小总估计成本时才检查节点是否包含在目标集中。如果节点current在目标集中,则heuristic\_cost\_to\_go(current)为零,并且由于所有边成本都是正的,我们知道将来找到的任何路径的成本必须大于或等于past\_cost[current]。因此,到current的路径必须是最短路径。(可能有其他相同成本的路径。)

如果启发式“成本到目标”是精确计算的,考虑障碍物,则$A^*$将从解决问题所需的最少节点数进行探索。当然,精确计算成本到目标等价于解决路径规划问题,所以这是不切实际的。相反,启发式成本到目标应该快速计算,并且应该尽可能接近实际成本到目标,以确保算法高效运行。使用乐观成本到目标确保最优解。

$A^*$算法是通用最佳优先搜索类的示例,它总是从当前根据某种度量被认为“最佳”的节点进行探索。算法10.1中描述了$A^*$搜索算法的伪代码。

\begin{algorithm}
\caption{$A^*$搜索}
\begin{algorithmic}[1]
\STATE OPEN $\leftarrow$ \{1\}
\STATE past\_cost[1] $\leftarrow$ 0, past\_cost[node] $\leftarrow$ infinity for node $\in$ \{2, \ldots, N\}
\WHILE{OPEN is not empty}
    \STATE current $\leftarrow$ first node in OPEN, remove from OPEN
    \STATE add current to CLOSED
    \IF{current is in the goal set}
        \RETURN SUCCESS and the path to current
    \ENDIF
    \FOR{each nbr of current not in CLOSED}
        \STATE tentative\_past\_cost $\leftarrow$ past\_cost[current] + cost[current,nbr]
        \IF{tentative\_past\_cost < past\_cost[nbr]}
            \STATE past\_cost[nbr] $\leftarrow$ tentative\_past\_cost
            \STATE parent[nbr] $\leftarrow$ current
            \STATE put (or move) nbr in sorted list OPEN according to est\_total\_cost[nbr] $\leftarrow$ past\_cost[nbr] + heuristic\_cost\_to\_go(nbr)
        \ENDIF
    \ENDFOR
\ENDWHILE
\RETURN FAILURE
\end{algorithmic}
\end{algorithm}

\subsubsection{其他搜索方法}

\begin{itemize}
\item \textbf{Dijkstra算法。}如果启发式成本到目标总是估计为零,则$A^*$总是从已以最小过去成本到达的OPEN节点进行探索。这种变体称为Dijkstra算法,它在历史上先于$A^*$。Dijkstra算法也保证找到最小成本路径,但在许多问题上由于缺乏启发式前瞻函数来帮助引导搜索,它运行得比$A^*$慢。

\item \textbf{广度优先搜索。}如果$E$中的每条边具有相同的成本,Dijkstra算法简化为广度优先搜索。首先考虑距离起始节点一条边的所有节点,然后考虑距离两条边的所有节点,等等。因此,找到的第一个解是最小成本路径。

\item \textbf{次优$A^*$搜索。}如果启发式成本到目标通过将乐观启发式乘以常数因子$\eta > 1$而被高估,则$A^*$搜索将偏向于从更接近目标的节点进行探索,而不是具有低过去成本的节点。这可能导致更快找到解,但与乐观成本到目标启发式的情况不同,解不保证是最优的。一种可能性是运行具有膨胀成本到目标的$A^*$以找到初始解,然后使用逐渐较小的$\eta$值重新运行搜索,直到分配的搜索时间已过或找到$\eta = 1$的解。
\end{itemize}

\section{完备路径规划器}

完备路径规划器依赖于自由C空间$C_{\text{free}}$的精确表示。这些技术往往在数学和算法上都很复杂,对于许多实际系统来说是不切实际的,因此我们不详细讨论它们。

完备路径规划的一种方法,我们将在第10.5节的修改形式中看到,基于用具有以下性质的一维路线图$R$表示复杂的高维空间$C_{\text{free}}$:

\begin{enumerate}
\item \textbf{可达性。}从$C_{\text{free}}$中的每个点$q$,可以轻松找到到$R$中点$q'$的自由路径(例如,直线路径)。
\item \textbf{连通性。}对于$C_{\text{free}}$的每个连通分量,$R$有一个连通分量。
\end{enumerate}

有了这样的路线图,规划器可以通过简单地找到从$q_{\text{start}}$到路线图上的点$q'_{\text{start}} \in R$的路径,从路线图上的点$q'_{\text{goal}} \in R$到$q_{\text{goal}}$的路径,以及沿着路线图$R$从$q'_{\text{start}}$到$q'_{\text{goal}}$的路径,来找到$C_{\text{free}}$的同一连通分量中任意两点$q_{\text{start}}$和$q_{\text{goal}}$之间的路径。如果可以轻松找到$q_{\text{start}}$和$q_{\text{goal}}$之间的路径,则可能甚至不使用路线图。

虽然构建$C_{\text{free}}$的路线图在一般情况下是复杂的,但一些问题允许简单的路线图。例如,考虑在平面中的多边形障碍物之间平移的多边形机器人。如图10.5所示,在这种情况下C障碍物也是多边形。合适的路线图是加权无向可见性图,节点在C障碍物的顶点处,边在可以“看到”彼此的节点之间(即,顶点之间的线段不与障碍物相交)。与每条边相关的权重是节点之间的欧几里得距离。

这不仅是一个合适的路线图$R$,而且它允许我们使用$A^*$搜索来找到$C_{\text{free}}$的同一连通分量中任意两个配置之间的最短路径,因为最短路径保证要么是从$q_{\text{start}}$到$q_{\text{goal}}$的直线,要么由从$q_{\text{start}}$到节点$q'_{\text{start}} \in R$的直线、从节点$q'_{\text{goal}} \in R$到$q_{\text{goal}}$的直线,以及沿着$R$的直边从$q'_{\text{start}}$到$q'_{\text{goal}}$的路径组成(图10.9)。注意最短路径要求机器人掠过障碍物,因此我们隐式地将$C_{\text{free}}$视为包括其边界。

\section{网格方法}

像$A^*$这样的搜索算法需要搜索空间的离散化。C空间最简单的离散化是网格。例如,如果配置空间是$n$维的,我们希望在每个维度上有$k$个网格点,则C空间由$k^n$个网格点表示。

$A^*$算法可以用作C空间网格的路径规划器,进行以下小幅修改:

\begin{itemize}
\item 必须选择网格点的“邻居”的定义:机器人是否被约束在配置空间中沿轴对齐方向移动,或者它是否可以同时在多个维度中移动?例如,对于二维C空间,邻居可以是4连通的(在罗盘的基本点上:北、南、东和西)或8连通的(允许对角线),如图10.10(a)所示。如果允许对角线运动,应该适当地惩罚对角线邻居的成本。例如,到北、南、东或西邻居的成本可以是1,而到对角线邻居的成本可以是$\sqrt{2}$。如果需要整数以提高实现的效率,可以使用近似成本5和7。
\item 如果只使用轴对齐运动,启发式成本到目标应该基于曼哈顿距离,而不是欧几里得距离。曼哈顿距离计算必须行驶的“城市街区”数量,规则是块的对角线是不可能的(图10.10(b))。
\item 只有当从current到nbr的步骤是无碰撞时,才将节点nbr添加到OPEN。(如果nbr处的增长版本的机器人不与任何障碍物相交,则可以认为该步骤是无碰撞的。)
\item 由于网格的已知规则结构,可以进行其他优化。
\end{itemize}

基于$A^*$网格的路径规划器是分辨率完备的:如果解在C空间的离散化级别存在,它将找到解。路径将是允许运动的最短路径。

图10.10(c)说明了图10.2的2R机器人示例的基于网格的路径规划。C空间表示为$k = 32$的网格,即每个关节的分辨率为$360^\circ / 32 = 11.25^\circ$。这总共产生$32^2 = 1024$个网格点。

如上所述,基于网格的规划器是单查询规划器:它从头开始解决每个路径规划查询。但是,如果相同的$q_{\text{goal}}$将在同一环境中用于多个路径规划查询,可能值得预处理整个网格以启用快速路径规划。这是波前规划器,如图10.11所示。

虽然基于网格的路径规划易于实现,但它仅适用于低维C空间。原因是网格点的数量,以及因此路径规划器的计算复杂度,随维度数$n$呈指数增长。例如,在$n = 3$维的C空间中,分辨率$k = 100$导致$k^n = 100$万个网格节点,而$n = 5$导致100亿个节点,$n = 7$导致100万亿个节点。另一种选择是沿每个维度减少分辨率$k$,但这导致C空间的粗糙表示,可能错过自由路径。

\subsection{多分辨率网格表示}

减少基于网格的规划器计算复杂度的一种方法是使用$C_{\text{free}}$的多分辨率网格表示。概念上,如果以网格点为中心的矩形单元的任何部分接触C障碍物,则认为网格点是障碍物。为了细化障碍物的表示,可以将障碍物单元细分为更小的单元。原始单元的每个维度被分成两半,对于$n$维空间产生$2^n$个子单元。仍然与C障碍物接触的任何单元然后进一步细分,直到指定的最大分辨率。

这种表示的优势是只有C空间中靠近障碍物的部分被细化到高分辨率,而那些远离障碍物的部分由粗糙分辨率表示。这允许规划器在杂乱空间中通过短步骤找到路径,同时在开阔空间中采取大步骤。这个想法在图10.12中说明,它仅使用10个单元来表示障碍物,与使用64个单元的固定网格具有相同的分辨率。

对于$n = 2$,这种多分辨率表示称为四叉树,因为每个障碍物单元细分为$2^n = 4$个单元。对于$n = 3$,每个障碍物单元细分为$2^n = 8$个单元,表示称为八叉树。

$C_{\text{free}}$的多分辨率表示可以在搜索之前构建,或者在执行搜索时增量构建。在后一种情况下,如果发现从current到nbr的步骤碰撞,可以将步骤大小减半,直到步骤是自由的或达到最小步骤大小。

\subsection{具有运动约束的网格方法}

上述基于网格的规划器在机器人可以从规则C空间网格中的一个单元移动到任何相邻单元的假设下运行。这对于某些机器人可能是不可能的。例如,汽车无法一步到达其侧面的“邻居”单元。此外,对于快速移动的机械臂,应该在状态空间中规划运动,而不仅仅是在C空间中,以考虑臂的动力学。在状态空间中,机器人无法在 certain 方向上移动(图10.13)。

基于网格的规划器必须适应特定机器人的运动约束。特别是,约束可能导致有向图。一种方法是在仍然利用C空间或状态空间(如适用)上的网格的同时离散化机器人控制。下面描述了轮式移动机器人和动态机械臂的细节。

\subsubsection{轮式移动机器人的基于网格路径规划}

如第13.3节所述,独轮车、差动驱动和类车机器人的简化模型的控制是$(v, \omega)$,即前后线性速度和角速度。这些移动机器人的控制集如图10.14所示。还显示了建议的控制离散化,作为点。可以选择其他离散化。

使用控制离散化,我们可以使用Dijkstra算法的变体来找到短路径(算法10.2)。

搜索从$q_{\text{start}}$扩展,通过向前积分每个控制时间$\Delta t$,为无碰撞的路径创建新节点。每个节点跟踪用于到达该节点的控制以及到达该节点的路径成本。到达新节点的路径成本是前一个节点current的成本加上动作的成本的总和。

控制的积分不会将移动机器人移动到精确的网格点。相反,C空间网格在第9行和第10行发挥作用。当扩展节点时,它所在的网格单元被标记为"已占用"。随后,此已占用单元中的任何节点将从搜索中剪枝。这防止搜索扩展那些已经被以更低成本到达的节点附近的节点。在搜索期间考虑不超过MAXCOUNT个节点,其中MAXCOUNT是用户选择的值。

时间$\Delta t$应该选择得使每个运动步骤都"小"。网格单元的大小应该选择得尽可能大,同时确保任何控制积分时间$\Delta t$都会将移动机器人移出其当前网格单元。

当current位于目标区域内,或者没有更多节点要扩展(可能因为障碍物),或者已考虑MAXCOUNT个节点时,规划器终止。对于成本函数和问题的其他参数的选择,找到的任何路径都是最优的。规划器实际上在有些杂乱的空间中运行得更快,因为障碍物有助于引导探索。

图10.15显示了一些汽车运动规划的例子。

\subsubsection{机械臂的基于网格运动规划}

规划机械臂运动的一种方法是将问题解耦为路径规划问题,然后对路径进行时间标定:
\begin{enumerate}
\item[(a)] 应用基于网格或其他路径规划器在C空间中找到无障碍路径。
\item[(b)] 对路径进行时间标定,找到尊重机器人动力学的最快轨迹,如第9.4节所述,或使用任何不太激进的时间标定。
\end{enumerate}

由于运动规划问题被分为两个步骤(路径规划后跟时间标定),所得运动通常不是时间最优的。

另一种方法是直接在状态空间中规划。给定机械臂的状态$(q, \dot{q})$,设$A(q, \dot{q})$表示基于有限关节力矩可行的加速度集合。为了离散化控制,集合$A(q, \dot{q})$与形式为
\begin{equation}
\sum_{i=1}^n c a_i \hat{e}_i
\end{equation}
的点的网格相交,其中$c$是整数,$a_i > 0$是$\ddot{q}_i$方向的加速度步长,$\hat{e}_i$是第$i$方向的单位向量(图10.16)。

当机器人移动时,加速度集合$A(q, \dot{q})$改变,但网格保持固定。因此,假设在运动规划中的每个"步骤"处固定积分时间$\Delta t$,机器人的可达状态(在任何整数步数之后)被限制在状态空间中的网格上。为了看到这一点,考虑机器人的单个关节角度$q_1$,并为了简单起见假设零初始速度,$\dot{q}_1(0) = 0$。时间步$k$处的速度取形式
\begin{equation}
\dot{q}_1(k) = \dot{q}_1(k-1) + c(k) a_1 \Delta t
\end{equation}
其中$c(k)$是有限整数集合中的任何值。通过归纳,任何时间步的速度必须是形式$a_1 k_v \Delta t$,其中$k_v$是整数。时间步$k$处的位置取形式
\begin{equation}
q_1(k) = q_1(k-1) + \dot{q}_1(k-1)\Delta t + \frac{1}{2} c(k) a_1 (\Delta t)^2
\end{equation}

为了找到从起始节点到目标集的轨迹,可以使用广度优先搜索在状态空间节点上创建搜索树。当从状态空间中的节点$(q, \dot{q})$进行探索时,评估集合$A(q, \dot{q})$以找到离散控制动作集合。通过积分控制动作时间$\Delta t$创建新节点。如果到达它的路径碰撞或它先前已被到达(即,通过花费相同或更少时间的轨迹),则丢弃节点。

因为关节角度和角速度是有界的,状态空间网格是有限的,因此可以在有限时间内搜索。规划器是分辨率完备的,并返回时间最优轨迹,受控于控制网格和时间步$\Delta t$中指定的分辨率。

控制网格步长$a_i$必须选择得足够小,使得对于任何可行状态$(q, \dot{q})$,$A(q, \dot{q})$包含控制网格的代表性点集合。为控制选择更细的网格,或更小的时间步$\Delta t$,在状态空间中创建更细的网格,并在障碍物中找到解的可能性更高。它还允许选择更小的目标集,同时保持状态空间网格的点在集合内。

更细的离散化需要计算成本。如果控制离散化的分辨率在每个维度上增加因子$r$(即,每个$a_i$减少到$a_i/r$),并且时间步大小除以因子$\tau$,则对于给定的机器人运动持续时间,增长搜索树所花费的计算时间增加因子$r^{n\tau}$,其中$n$是关节数。例如,对于三关节机器人,将控制网格分辨率增加因子$r = 2$并将时间步减少因子$\tau = 4$,导致搜索可能需要$2^{3 \times 4} = 4096$倍的时间来完成。规划器的高计算复杂度使其在超过几个自由度时变得不切实际。

上面的描述忽略了一个重要问题:可行控制集合$A(q, \dot{q})$在时间步期间改变,因此在时间步开始时选择的控制在时间步结束时可能不再可行。因此,应该使用保守近似$\tilde{A}(q, \dot{q}) \subset A(q, \dot{q})$代替。无论选择哪个控制动作,该集合应该在时间步的持续时间内保持可行。如何确定这样的保守近似$\tilde{A}(q, \dot{q})$超出了本章的范围,但它与臂的质量矩阵$M(q)$随$q$变化的速度以及机器人移动的速度的界限有关。在低速$\dot{q}$和短持续时间$\Delta t$时,保守集合$\tilde{A}(q, \dot{q})$非常接近$A(q, \dot{q})$。

\section{采样方法}

上面讨论的每种基于网格的方法在所选离散化下提供最优解。然而,这些方法的一个缺点是它们的高计算复杂度,使它们不适合具有超过几个自由度的系统。

一类不同的规划器,称为采样方法,依赖于随机或确定性函数从C空间或状态空间中选择样本:评估样本或运动是否在$X_{\text{free}}$中的函数;确定附近先前自由空间样本的函数;以及尝试从先前样本连接到或移向新样本的简单局部规划器。这些函数用于构建表示机器人可行运动的图或树。

采样方法通常放弃网格搜索的分辨率最优解,以换取在高维状态空间中快速找到满意解的能力。选择样本以形成路线图或搜索树,该树使用比通常由固定高分辨率网格所需的更少的样本快速近似自由空间$X_{\text{free}}$,其中网格点的数量随搜索空间的维度呈指数增长。大多数采样方法是概率完备的:当存在解时,找到解的概率随着样本数量趋于无穷而趋于100\%。

采样方法的两个主要类别是快速探索随机树(RRT)和概率路线图(PRM)。前者使用树表示在C空间或状态空间中进行单查询规划,而PRM主要是创建用于多查询规划的路线图图的C空间规划器。

\subsection{RRT算法}

RRT算法搜索从初始状态$x_{\text{start}}$到目标集$X_{\text{goal}}$的无碰撞运动。它应用于运动学问题,其中状态$x$简单地是配置$q$,以及动力学问题,其中状态包括速度。基本RRT从$x_{\text{start}}$增长单个树,如算法10.3所述。

\begin{algorithm}
\caption{RRT算法}
\begin{algorithmic}[1]
\STATE 用$x_{\text{start}}$初始化搜索树$T$
\WHILE{$T$小于最大树大小}
    \STATE $x_{\text{samp}} \leftarrow$ 从$X$采样
    \STATE $x_{\text{nearest}} \leftarrow$ $T$中距离$x_{\text{samp}}$最近的节点
    \STATE 使用局部规划器找到从$x_{\text{nearest}}$到$x_{\text{new}}$的运动,方向朝向$x_{\text{samp}}$
    \IF{运动是无碰撞的}
        \STATE 将$x_{\text{new}}$添加到$T$,边从$x_{\text{nearest}}$到$x_{\text{new}}$
        \IF{$x_{\text{new}}$在$X_{\text{goal}}$中}
            \RETURN SUCCESS和到$x_{\text{new}}$的运动
        \ENDIF
    \ENDIF
\ENDWHILE
\RETURN FAILURE
\end{algorithmic}
\end{algorithm}

在运动学问题的典型实现中(其中$x$简单地是$q$),第3行中的采样器从$X$上的几乎均匀分布中随机选择$x_{\text{samp}}$,对$X_{\text{goal}}$中的状态有轻微偏向。搜索树$T$中最近的节点$x_{\text{nearest}}$(第4行)是使到$x_{\text{samp}}$的欧几里得距离最小的节点。状态$x_{\text{new}}$(第5行)被选择为从$x_{\text{nearest}}$到$x_{\text{samp}}$的直线上距离$d$小的状态。因为$d$很小,非常简单的局部规划器,例如返回直线运动的规划器,通常会找到连接$x_{\text{nearest}}$到$x_{\text{new}}$的运动。如果运动是无碰撞的,新状态$x_{\text{new}}$被添加到搜索树$T$。

净效果是几乎均匀分布的样本"拉"树朝向它们,导致树快速探索$X_{\text{free}}$。图10.17显示了这种拉动动作对探索的影响的示例。

基本算法为程序员留下了许多选择:如何从$X$采样(第3行),如何定义$T$中的"最近"节点(第4行),以及如何规划运动以朝向$x_{\text{samp}}$取得进展(第5行)。即使对采样方法的微小改变,例如,也可能导致规划器运行时间的显著变化。基于这些选择和其他变体,文献中已经提出了各种各样的规划器。下面描述其中一些变体。

\subsubsection{第3行:采样器}

最明显的采样器是从$X$上的均匀分布中随机采样的采样器。这对于欧几里得C空间$\mathbb{R}^n$以及$n$关节机器人C空间$\mathcal{T}^n = S^1 \times \cdots \times S^1$($n$次)是直接的,其中我们可以为每个关节角度选择均匀分布,以及对于平面中移动机器人的C空间$\mathbb{R}^2 \times S^1$,其中我们可以分别为$\mathbb{R}^2$和$S^1$选择均匀分布。在某些其他弯曲C空间(例如$SO(3)$)上的均匀分布的概念不太直接。

对于动力学系统,状态空间上的均匀分布可以定义为C空间上的均匀分布和有界速度集合上的均匀分布的叉积。

尽管名称"快速探索随机树"来自随机采样策略的想法,实际上样本不需要随机生成。例如,可以改用确定性采样方案,该方案在$X$上生成逐渐更细的(多分辨率)网格。为了反映这种更一般的观点,该方法被称为快速探索密集树(RDT),强调样本应该最终在状态空间中变得密集的关键点(即,随着样本数量趋于无穷,样本变得任意接近$X$中的每个点)。

\subsubsection{第4行:定义最近节点}

找到"最近"节点取决于$X$上距离的定义。对于$C = \mathbb{R}^n$上的无约束运动学机器人,两点之间距离的自然选择简单地是欧几里得距离。对于其他空间,选择不太明显。

例如,对于具有C空间$\mathbb{R}^2 \times S^1$的类车机器人,哪个配置最接近配置$x_{\text{samp}}$:相对于$x_{\text{samp}}$旋转20度的配置,在其正后方2米的配置,还是在其正侧1米的配置(图10.18)?由于运动约束防止原地旋转或直接侧向移动,正后方2米的配置最适合朝向$x_{\text{samp}}$取得进展。因此定义距离概念需要:
\begin{itemize}
\item 将不同单位的组件(例如,度、米、度/秒、米/秒)组合成单一距离度量;以及
\item 考虑机器人的运动约束。
\end{itemize}

最近的节点$x_{\text{nearest}}$可能应该定义为能够最快到达$x_{\text{samp}}$的节点,但计算这与解决运动规划问题一样困难。

从$x$到$x_{\text{samp}}$的距离度量的简单选择是沿$x_{\text{samp}} - x$的不同组件的距离的加权和。权重表示不同组件的相对重要性。如果更多地了解机器人可以从状态$x$在有限时间内到达的状态集合,这些信息可以用于确定最近节点。无论如何,应该快速计算最近节点。找到最近邻是计算几何中的常见问题,可以使用各种算法(如kd树和哈希)高效解决。

\subsubsection{第5行:局部规划器}

局部规划器的工作是找到从$x_{\text{nearest}}$到某个点$x_{\text{new}}$的运动,该点更接近$x_{\text{samp}}$。规划器应该简单且应该快速运行。以下是三个示例。

\textbf{直线规划器。}计划是到$x_{\text{new}}$的直线,可以在$x_{\text{samp}}$处选择,或者在从$x_{\text{nearest}}$到$x_{\text{samp}}$的直线上距离$d$处选择。这适用于没有运动约束的运动学系统。

\textbf{离散化控制规划器。}对于具有运动约束的系统,例如轮式移动机器人或动力学系统,可以将控制离散化为离散集合$\{u_1, u_2, \ldots\}$,如具有运动约束的网格方法(第10.4.2节和图10.14和图10.16)。每个控制从$x_{\text{nearest}}$使用$\dot{x} = f(x, u)$积分固定时间$\Delta t$。在没有碰撞到达的新状态中,选择最接近$x_{\text{samp}}$的状态作为$x_{\text{new}}$。

\textbf{轮式机器人规划器。}对于轮式移动机器人,可以使用Reeds–Shepp曲线找到局部计划,如第13.3.3节所述。

可以设计其他特定于机器人的局部规划器。

\subsubsection{其他RRT变体}

基本RRT算法的性能在很大程度上取决于采样方法、距离度量和局部规划器的选择。除了这些选择之外,下面概述了基本RRT的其他两个变体。

\textbf{双向RRT。}双向RRT增长两棵树:一棵从$x_{\text{start}}$"向前",一棵从$x_{\text{goal}}$"向后"。算法在增长前向树和增长后向树之间交替,并且经常尝试通过从另一棵树选择$x_{\text{samp}}$来连接两棵树。这种方法的优点是可以精确到达单个目标状态$x_{\text{goal}}$,而不仅仅是目标集$X_{\text{goal}}$。另一个优点是在许多环境中,两棵树可能比单个"前向"树找到目标集更快地找到彼此。

主要问题是局部规划器可能无法精确连接两棵树。例如,第10.5.1.3节的离散化控制规划器极不可能创建精确到另一棵树中节点的运动。在这种情况下,当每棵树上的点足够接近时,两棵树可能被认为或多或少连接。可以返回"断裂"的不连续轨迹,并通过平滑方法(第10.8节)进行修补。

\textbf{RRT$^*$。}基本RRT算法一旦找到到$X_{\text{goal}}$的运动就返回SUCCESS。另一种方法是继续运行算法,并且仅在达到另一个终止条件(例如,最大运行时间或最大树大小)时终止搜索。然后可以返回具有最小成本的运动。这样,RRT解可能随着时间推移继续改进。然而,因为树中的边永远不会被删除或更改,RRT通常不会收敛到最优解。

RRT$^*$算法是单树RRT的变体,它不断重新布线搜索树,以确保它始终编码从$x_{\text{start}}$到树中每个节点的最短路径。基本方法适用于没有运动约束的C空间路径规划,允许从任何节点到任何其他节点的精确路径。

为了将RRT修改为RRT$^*$,RRT算法的第7行(将$x_{\text{new}}$插入$T$,边从$x_{\text{nearest}}$到$x_{\text{new}}$)被替换为测试$T$中所有足够接近$x_{\text{new}}$的节点$x \in X_{\text{near}}$。从$x \in X_{\text{near}}$创建到$x_{\text{new}}$的边,由局部规划器(1)具有无碰撞运动且(2)最小化从$x_{\text{start}}$到$x_{\text{new}}$的路径的总成本,而不仅仅是添加边的成本。总成本是到达候选$x \in X_{\text{near}}$的成本加上新边的成本。

下一步是考虑每个$x \in X_{\text{near}}$,看看是否可以通过通过$x_{\text{new}}$的运动以更低的成本到达它。如果是,$x$的父节点更改为$x_{\text{new}}$。这样,树被增量重新布线,以消除高成本运动,支持到目前为止可用的最小成本运动。

$X_{\text{near}}$的定义取决于树中的样本数量、采样方法的细节、搜索空间的维度和其他因素。

与RRT不同,RRT$^*$提供的解随着样本节点数量的增加而接近最优解。与RRT一样,RRT$^*$算法是概率完备的。图10.19演示了RRT$^*$与RRT相比在$C = \mathbb{R}^2$的简单示例中的重新布线行为。

\subsection{PRM算法}

PRM使用采样在回答任何特定查询之前构建$C_{\text{free}}$的路线图表示(第10.3节)。路线图是无向图:机器人可以沿任何边在两个方向上精确地从一个节点移动到下一个节点。因此,PRM主要适用于运动学问题,对于这些问题存在精确的局部规划器,可以找到从任何$q_1$到任何其他$q_2$的路径(忽略障碍物)。最简单的例子是没有运动学约束的机器人的直线规划器。

一旦构建了路线图,可以通过尝试将其连接到路线图(从最近的节点开始)将特定的起始节点$q_{\text{start}}$添加到图中。对目标节点$q_{\text{goal}}$执行相同的操作。然后通常使用$A^*$搜索图中的路径。因此,一旦构建了路线图,可以高效地回答查询。

PRM的使用允许相对于使用高分辨率网格表示构建路线图,快速高效地构建路线图的可能性。原因是从给定配置由局部规划器"可见"的C空间的体积分数通常不会随C空间维度的增加而呈指数下降。

构建具有$N$个节点的路线图$R$的算法在算法10.4中概述,并在图10.20中说明。

\begin{algorithm}
\caption{PRM路线图构建算法(无向图)}
\begin{algorithmic}[1]
\FOR{$i = 1, \ldots, N$}
    \STATE $q_i \leftarrow$ 从$C_{\text{free}}$采样
    \STATE 将$q_i$添加到$R$
\ENDFOR
\FOR{$i = 1, \ldots, N$}
    \STATE $N(q_i) \leftarrow$ $q_i$的$k$个最近邻居
    \FOR{每个$q \in N(q_i)$}
        \IF{存在从$q$到$q_i$的无碰撞局部路径且尚未存在从$q$到$q_i$的边}
            \STATE 将边从$q$到$q_i$添加到路线图$R$
        \ENDIF
    \ENDFOR
\ENDFOR
\RETURN $R$
\end{algorithmic}
\end{algorithm}

PRM路线图构建算法中的关键选择是如何从$C_{\text{free}}$采样。虽然默认可能是从$C$上的均匀分布随机采样并消除碰撞中的配置,但已经表明在障碍物附近更密集地采样可以提高找到窄通道的可能性,从而显著减少正确表示$C_{\text{free}}$连通性所需的样本数量。另一种选择是确定性多分辨率采样。

\section{虚拟势场}

虚拟势场方法受到自然界中势能场的启发,例如重力和磁场。从物理学我们知道,在$C$上定义的势场$P(q)$产生力$F = -\partial P/\partial q$,该力将物体从高势能驱动到低势能。例如,如果在均匀重力势场($g = 9.81$ m/s$^2$)中$h$是地球表面以上的高度,则质量$m$的势能是$P(h) = mgh$,作用在其上的力是$F = -\partial P/\partial h = -mg$。该力将导致质量落到地球表面。

在机器人运动控制中,目标配置$q_{\text{goal}}$被分配低虚拟势能,障碍物被分配高虚拟势能。对机器人施加与虚拟势能的负梯度成比例的力自然地将机器人推向目标并远离障碍物。

虚拟势场与我们到目前为止看到的规划器非常不同。通常可以快速计算场的梯度,因此可以实时计算运动(反应控制),而不是提前规划。使用适当的传感器,该方法甚至可以处理移动或意外出现的障碍物。基本方法的缺点是机器人可能卡在势场的局部最小值中,远离目标,即使存在到目标的可行运动。在某些情况下,可以设计势能以保证唯一的局部最小值在目标处,从而消除这个问题。

\subsection{C空间中的点}

让我们首先假设C空间中的点机器人。目标配置$q_{\text{goal}}$通常由在目标处具有零能量的二次势能"碗"编码:
\begin{equation}
P_{\text{goal}}(q) = \frac{1}{2}(q - q_{\text{goal}})^T K(q - q_{\text{goal}})
\end{equation}
其中$K$是对称正定加权矩阵(例如,单位矩阵)。该势能产生的力是
\begin{equation}
F_{\text{goal}}(q) = -\frac{\partial P_{\text{goal}}}{\partial q} = K(q_{\text{goal}} - q)
\end{equation}
与到目标的距离成比例的吸引力。

由C障碍物$B$引起的排斥势能可以从到障碍物的距离$d(q, B)$计算(第10.2.2节):
\begin{equation}
P_B(q) = \frac{k}{2d^2(q, B)}
\end{equation}
其中$k > 0$是缩放因子。势能仅对障碍物外的点正确定义,$d(q, B) > 0$。由障碍物势能产生的力是
\begin{equation}
F_B(q) = -\frac{\partial P_B}{\partial q} = \frac{k}{d^3(q, B)} \frac{\partial d}{\partial q}
\end{equation}

总势能通过将吸引目标势能和排斥障碍物势能相加获得:
\begin{equation}
P(q) = P_{\text{goal}}(q) + \sum_i P_{B_i}(q)
\end{equation}
产生总力
\begin{equation}
F(q) = F_{\text{goal}}(q) + \sum_i F_{B_i}(q)
\end{equation}

注意吸引和排斥势能的总和可能不会在$q_{\text{goal}}$处给出最小值(零力)。此外,通常对最大势能和力设置界限,因为简单的障碍物势能(10.3)否则会在障碍物边界附近产生无界势能和力。

图10.21显示了$\mathbb{R}^2$中具有三个圆形障碍物的点的势场。势场的等高线图清楚地显示了空间中心附近的全局最小值(接近标记为$+$的目标)、左侧两个障碍物附近的局部最小值,以及障碍物附近的鞍点(在一个方向上是最大值,在另一个方向上是最小值的临界点)。鞍点通常不是问题,因为小的扰动允许继续朝向目标取得进展。然而,远离目标的局部最小值是一个问题,因为它们吸引附近的状态。

为了使用计算的$F(q)$实际控制机器人,我们有几个选项,其中两个是:
\begin{itemize}
\item 应用计算的力加上阻尼:
\begin{equation}
u = F(q) + B\dot{q}
\end{equation}
如果$B$是正定的,则它对所有$\dot{q} \neq 0$耗散能量,减少振荡并保证机器人将静止。如果$B = 0$,机器人在保持恒定总能量的同时继续移动,这是初始动能$\frac{1}{2}\dot{q}^T(0)M(q(0))\dot{q}(0)$和初始虚拟势能$P(q(0))$的总和。

在控制律(10.4)下机器人的运动可以可视化为球在图10.21的势面上在重力作用下滚动,其中耗散力是滚动摩擦。

\item 将计算的力视为命令速度:
\begin{equation}
\dot{q} = F(q)
\end{equation}
这自动消除振荡。
\end{itemize}

使用简单的障碍物势能(10.3),即使远处的障碍物对机器人的运动也有非零影响。为了加速排斥项的评估,可以忽略远处的障碍物。我们可以定义障碍物的影响范围$d_{\text{range}} > 0$,使得对于所有$d(q, B) \geq d_{\text{range}}$,势能为零:
\begin{equation}
U_B(q) = \begin{cases}
\frac{k}{2} \left(\frac{d_{\text{range}} - d(q, B)}{d_{\text{range}} d(q, B)}\right)^2 & \text{如果 } d(q, B) < d_{\text{range}} \\
0 & \text{否则}
\end{cases}
\end{equation}

另一个问题是$d(q, B)$及其梯度通常难以计算。处理这个问题的方法在第10.6.3节中描述。

\subsection{导航函数}

势场方法的一个显著问题是局部最小值。虽然势场可能适用于相对不杂乱的空间或对意外障碍物的快速响应,但它们可能使机器人在许多实际应用中卡在局部最小值中。

避免这个问题的一种方法是图10.11的波前规划器。波前算法通过在自由空间的网格表示中从目标单元可到达的每个单元的广度优先遍历,创建无局部最小值的势函数。因此,如果存在运动规划问题的解,则保证简单地每步"下坡"移动将机器人带到目标。

另一种无局部最小值梯度跟随方法基于用导航函数替换虚拟势函数。导航函数$\phi(q)$是一种虚拟势函数,它:
\begin{enumerate}
\item 在$q$上是光滑的(或至少二次可微);
\item 在所有障碍物的边界上具有有界最大值(例如,1);
\item 在$q_{\text{goal}}$处具有单个最小值;以及
\item 在所有临界点$q$(其中$\partial \phi/\partial q = 0$)处具有满秩Hessian$\partial^2 \phi/\partial q^2$(即,$\phi(q)$是Morse函数)。
\end{enumerate}

条件1确保Hessian$\partial^2 \phi/\partial q^2$存在。条件2对机器人的虚拟势能设置上界。关键条件是3和4。条件3确保$\phi(q)$的临界点(包括最小值、最大值和鞍点)中,只有一个最小值,在$q_{\text{goal}}$处。这确保$q_{\text{goal}}$至少是局部吸引的。可能存在沿方向子集是最小值的鞍点,但条件4确保被任何鞍点吸引的初始状态集合具有空内部(零测度),因此几乎每个初始状态收敛到唯一最小值$q_{\text{goal}}$。

虽然构建只有单个最小值的导航势函数不是平凡的,[152]展示了如何为特定情况构建它们:$n$维$C_{\text{free}}$由半径$R$的$n$球内和以$q_i$为中心、半径为$r_i$的较小球形障碍物$B_i$外的所有点组成,即$\{q \in \mathbb{R}^n | \|q\| \leq R \text{且} \|q - q_i\| > r_i \text{对所有} i\}$。这称为球世界。虽然真实的C空间不太可能是球世界,但Rimon和Koditschek表明障碍物的边界以及相关的导航函数可以变形为更广泛的星形障碍物类。星形障碍物是具有中心点的障碍物,从该中心点到障碍物边界上任何点的线段完全包含在障碍物内。星世界是具有星形障碍物的星形C空间。

因此,为任意星世界找到导航函数简化为为"模型"球世界找到导航函数,该球世界在星形障碍物的中心处具有中心,然后将该导航函数拉伸和变形为适合星世界的导航函数。Rimon和Koditschek给出了完成此操作的系统程序。

图10.22显示了导航函数从模型球世界到星世界的变形,对于情况$C \subset \mathbb{R}^2$。

\subsection{工作空间势能}

计算来自障碍物的排斥力的一个困难是获得到障碍物的距离$d(q, B)$。避免精确计算的一种方法是将障碍物的边界表示为点障碍物集合,并将机器人表示为小的控制点集合。设机器人上控制点$i$的笛卡尔位置写为$f_i(q) \in \mathbb{R}^3$,障碍物的边界点$j$为$c_j \in \mathbb{R}^3$。然后两点之间的距离是$\|f_i(q) - c_j\|$,控制点$i$处由于障碍物点$j$的势能是
\begin{equation}
P'_{ij}(q) = \frac{k}{2\|f_i(q) - c_j\|^2}
\end{equation}
产生控制点处的排斥力
\begin{equation}
F'_{ij}(q) = -\frac{\partial P'_{ij}}{\partial q} = \frac{k}{\|f_i(q) - c_j\|^4} \left(\frac{\partial f_i}{\partial q}\right)^T (f_i(q) - c_j) \in \mathbb{R}^3
\end{equation}

为了将线性力$F'_{ij}(q) \in \mathbb{R}^3$转换为作用在机械臂或移动机器人上的广义力$F_{ij}(q) \in \mathbb{R}^n$,我们首先找到雅可比$J_i(q) \in \mathbb{R}^{3 \times n}$,它将$\dot{q}$与控制点$f_i$的线性速度$\dot{f}_i$相关联:
\begin{equation}
\dot{f}_i = \frac{\partial f_i}{\partial q} \dot{q} = J_i(q) \dot{q}
\end{equation}

根据虚功原理,由于排斥线性力$F'_{ij}(q) \in \mathbb{R}^3$产生的广义力$F_{ij}(q) \in \mathbb{R}^n$简单地是
\begin{equation}
F_{ij}(q) = J_i^T(q) F'_{ij}(q)
\end{equation}

现在作用在机器人上的总力$F(q)$是容易计算的吸引力$F_{\text{goal}}(q)$和所有$i$和$j$的排斥力$F_{ij}(q)$的总和。

\subsubsection{轮式移动机器人}

前面的分析假设控制力$u = F(q) + B\dot{q}$(控制律(10.4))或速度$\dot{q} = F(q)$(控制律(10.5))可以在任何方向上应用。然而,如果机器人是受滚动约束$A(q)\dot{q} = 0$的轮式移动机器人,则计算的$F(q)$必须投影到控制$F_{\text{proj}}(q)$,这些控制使机器人沿约束切向移动。对于采用控制律$\dot{q} = F_{\text{proj}}(q)$的运动学机器人,合适的投影是
\begin{equation}
F_{\text{proj}}(q) = \left[I - A^T(q)\left(A(q)A^T(q)\right)^{-1}A(q)\right] F(q)
\end{equation}

对于采用控制律$u = F_{\text{proj}}(q) + B\dot{q}$的动态机器人,投影在第8.7节中讨论。

\subsubsection{势场在规划器中的使用}

势场可以与路径规划器结合使用。例如,最佳优先搜索(如$A^*$)可以使用势能作为成本到目标的估计。结合搜索函数可以防止规划器永久卡在局部最小值中。

\section{非线性优化}

运动规划问题可以表示为具有等式和不等式约束的通用非线性优化,利用许多软件包来解决此类问题。非线性优化问题可以通过基于梯度的方法(如序列二次规划(SQP))或非梯度方法(如模拟退火、Nelder–Mead优化和遗传规划)来解决。像许多非线性优化问题一样,这些方法通常不保证在存在可行解时找到可行解,更不用说最优解了。然而,对于使用目标函数和约束梯度的 methods,如果我们从接近解的猜测开始过程,我们可以期望局部最优解。

一般问题可以写成如下形式:
\begin{align}
\text{找到} \quad & u(t), q(t), T \nonumber \\
\text{最小化} \quad & J(u(t), q(t), T) \label{eq:10.6} \\
\text{约束条件} \quad & \dot{x}(t) = f(x(t), u(t)), \quad \forall t \in [0, T] \label{eq:10.7} \\
& u(t) \in U, \quad \forall t \in [0, T] \label{eq:10.8} \\
& q(t) \in C_{\text{free}}, \quad \forall t \in [0, T] \label{eq:10.9} \\
& x(0) = x_{\text{start}} \label{eq:10.10} \\
& x(T) = x_{\text{goal}} \label{eq:10.11}
\end{align}

为了通过非线性优化近似解决这个问题,控制$u(t)$、轨迹$q(t)$以及等式和不等式约束(10.8)–(10.12)必须离散化。这通常通过确保约束在间隔$[0, T]$上均匀分布的固定数量的点处满足,并选择位置和/或控制历史的有限参数表示来完成。我们至少有三种选择如何参数化位置和控制:

\begin{enumerate}
\item[(a)] \textbf{参数化轨迹$q(t)$。}在这种情况下,我们直接求解轨迹参数。任何时间的控制$u(t)$使用运动方程计算。这种方法不适用于控制数少于配置变量的系统,$m < n$。

\item[(b)] \textbf{参数化控制$u(t)$。}我们直接求解$u(t)$。计算状态$x(t)$需要积分运动方程。

\item[(c)] \textbf{同时参数化$q(t)$和$u(t)$。}我们有更多的变量,因为我们同时参数化了$q(t)$和$u(t)$。此外,我们有更多的约束,因为$q(t)$和$u(t)$必须显式满足动力学方程$\dot{x} = f(x, u)$,通常在间隔$[0, T]$上均匀分布的固定数量的点处。我们必须小心选择$q(t)$和$u(t)$的参数化,使它们彼此一致,以便在这些点处可以满足动力学方程。
\end{enumerate}

轨迹或控制历史可以用多种方式参数化。参数可以是时间多项式的系数、截断傅里叶级数的系数、样条系数、小波系数、分段恒定加速度或力段等。例如,控制$u_i(t)$可以由时间多项式的$p + 1$个系数$a_j$表示:
\begin{equation}
u_i(t) = \sum_{j=0}^p a_j t^j
\end{equation}

除了状态或控制历史的参数外,总时间$T$可能是另一个控制参数。参数化的选择对在给定时间$t$计算$q(t)$和$u(t)$的效率有影响。它还确定状态和控制对参数的敏感性,以及每个参数是在所有时间$[0, T]$还是仅在有限时间支撑基上影响轮廓。这些是数值优化的稳定性和效率的重要因素。

\section{平滑}

网格规划器的轴对齐运动和采样规划器的随机运动可能导致机器人的急动运动。处理这个问题的一种方法是让规划器处理全局搜索解的工作,然后后处理所得运动以使其更平滑。

有许多方法可以做到这一点;下面概述两种可能性。

\textbf{非线性优化。}虽然基于梯度的非线性优化如果使用随机初始轨迹初始化可能无法找到解,但它可以成为有效的后处理步骤,因为计划用"合理"的解初始化优化。初始运动必须转换为控制的参数化表示,成本$J(u(t), q(t), T)$可以表示为$u(t)$或$q(t)$的函数。例如,成本函数
\begin{equation}
J = \frac{1}{2} \int_0^T \dot{u}^T(t)\dot{u}(t)dt
\end{equation}
惩罚快速变化的控制。这在人类运动控制中有类比,人类手臂运动的平滑性归因于关节处力矩变化率的最小化[188]。

\textbf{细分和重连。}局部规划器可用于尝试连接路径上两个远点。如果这个新连接是无碰撞的,它替换原始路径段。由于局部规划器设计为产生短而平滑的路径,新路径可能比原始路径更短且更平滑。这个测试和替换过程可以迭代应用于路径上随机选择的点。另一种可能性是使用递归过程,首先将路径细分为两部分,并尝试用更短的路径替换每一部分;然后,如果任一部分无法用更短的路径替换,它再次细分;依此类推。

\section{总结}

\begin{itemize}
\item 运动规划问题的相当一般的陈述如下。给定初始状态$x(0) = x_{\text{start}}$和期望的最终状态$x_{\text{goal}}$,找到时间$T$和控制集$u: [0, T] \to U$,使得运动满足$x(T) \in X_{\text{goal}}$且对所有$t \in [0, T]$有$q(x(t)) \in C_{\text{free}}$。

\item 运动规划问题可以分为以下类别:路径规划与运动规划;完全驱动与约束或欠驱动;在线与离线;最优与满意;精确与近似;有或无障碍物。

\item 运动规划器可以通过以下性质来表征:多查询与单查询;随时规划或不是;完备、分辨率完备、概率完备,或以上都不是;以及它们的计算复杂度程度。

\item 障碍物将C空间划分为自由C空间$C_{\text{free}}$和障碍空间$C_{\text{obs}}$,其中$C = C_{\text{free}} \cup C_{\text{obs}}$。障碍物可能将$C_{\text{free}}$分成不同的连通分量。在不同连通分量中的配置之间没有可行路径。

\item 配置$q$是否碰撞的保守检查使用机器人和障碍物的简化"增长"表示。如果增长体之间没有碰撞,则保证配置是无碰撞的。检查路径是否无碰撞通常涉及在精细间隔的点处对路径进行采样,并确保如果各个配置是无碰撞的,则机器人路径的扫过体积是无碰撞的。

\item C空间几何通常由节点和节点之间边的图表示,其中边表示可行路径。图可以是无向的(边在两个方向上流动)或有向的(边只在一个方向上流动)。边可以是未加权的或根据遍历成本加权的。树是没有循环的有向图,其中每个节点最多有一个父节点。

\item 路线图路径规划器使用$C_{\text{free}}$的图表示,路径规划问题可以使用从$q_{\text{start}}$到路线图的简单路径、沿着路线图的路径以及从路线图到$q_{\text{goal}}$的简单路径来解决。

\item $A^*$算法是一种流行的搜索方法,它在图上找到最小成本路径。它通过总是从(1)未探索的且(2)在具有最小估计总成本的路径上的节点进行探索来操作。估计总成本是从起始节点到达节点时遇到的边的权重之和,加上到目标的成本到目标的估计。为了确保搜索返回最优解,成本到目标的估计应该是乐观的。

\item 基于网格的路径规划器将C空间离散化为由规则网格上相邻点组成的图。可以使用多分辨率网格以允许在开阔空间中大步长,在障碍物边界附近小步长。

\item 离散化控制集允许具有运动约束的机器人利用基于网格的方法。如果积分控制不会将机器人精确地落在网格点上,如果同一网格单元中的状态已经以更低成本实现,则仍然可以剪枝新状态。

\item 基本RRT算法从$x_{\text{start}}$增长单个搜索树以找到到$X_{\text{goal}}$的运动。它依赖于采样器在$X$中找到样本$x_{\text{samp}}$,在搜索树中找到最近节点$x_{\text{nearest}}$的算法,以及从$x_{\text{nearest}}$找到朝向$x_{\text{samp}}$更近点的运动的局部规划器。选择采样以使树快速探索$X_{\text{free}}$。

\item 双向RRT从$x_{\text{start}}$和$x_{\text{goal}}$都增长搜索树,并尝试连接它们。RRT$^*$算法返回随着规划时间趋于无穷而趋于最优的解。

\item PRM为多查询规划构建$C_{\text{free}}$的路线图。路线图通过对$C_{\text{free}}$采样$N$次,然后使用局部规划器尝试将每个样本与其几个最近邻居连接来构建。使用$A^*$搜索路线图。

\item 虚拟势场受到势能场(如重力和电磁场)的启发。目标点产生吸引势能,而障碍物产生排斥势能。总势能$P(q)$是这些的总和,施加在机器人上的虚拟力是$F(q) = -\partial P/\partial q$。机器人通过施加这个力加上阻尼或通过模拟一阶动力学并用$F(q)$作为速度驱动机器人来控制。势场方法在概念上简单,但可能使机器人卡在远离目标的局部最小值中。

\item 导航函数是没有局部最小值的势函数。导航函数导致几乎全局收敛到$q_{\text{goal}}$。虽然它们通常难以设计,但可以为某些环境系统地设计它们。

\item 运动规划问题可以转换为具有等式和不等式约束的通用非线性优化问题。虽然优化方法可用于找到平滑的接近最优运动,但它们往往在杂乱的C空间中卡在局部最小值中。优化方法通常需要解的好的初始猜测。

\item 基于网格和基于采样的规划器返回的运动往往是急动的。使用非线性优化或细分和重连平滑计划可以改善运动质量。
\end{itemize}

\section{注释和参考文献}

广泛涵盖运动规划的优秀教科书包括Latombe在1991年的原始文本[80]以及Choset等人[27]和LaValle[83]的较新文本。运动规划最新技术的其他总结可以在《机器人手册》[70]中找到,特别是对于受非完整和驱动约束的机器人,在《控制手册》[101]、《系统和控制百科全书》[100]以及Murray等人的教科书[122]中。搜索算法和其他人工智能算法由Russell和Norvig[155]详细涵盖。

SRI的Shakey机器人运动规划的里程碑式早期工作导致Hart、Nilsson和Raphael在1968年开发了$A^*$搜索[53]。这项工作建立在Bellman和Dreyfus[10]描述的最优决策动态规划的新建立方法上,并改进了Dijkstra算法[37]的性能。$A^*$的次优随时变体在[90]中提出。基于C空间层次分解[156]的多分辨率路径规划的早期工作在[65, 96, 45, 54]中描述。

一条早期工作线专注于在存在障碍物的情况下精确表征自由C空间。多边形在多边形之间移动的可见性图方法由Lozano-Pérez和Wesley在1979年开发[97]。在更一般的设置中,研究人员使用复杂的算法和数学方法来开发自由C空间的单元分解和精确路线图。这项工作的重要例子是Schwartz和Sharir关于钢琴搬运工问题的一系列论文[159, 160, 161]以及Canny的博士论文[23]。

由于精确表示C空间拓扑所需的数学复杂性和高计算复杂度,在1990年代形成了使用样本近似表示C空间的运动,这一运动今天仍在继续。这条工作线遵循两个主要分支:概率路线图(PRM)[69]和快速探索随机树(RRT)[84, 86, 85]。由于它们能够相对高效地处理复杂的高维C空间,基于采样的规划器的研究已经爆炸式增长,一些后续工作在[27, 83]中总结。本章重点介绍的双向RRT和RRT$^*$分别在[83]和[68]中描述。

轮式移动机器人运动规划的基于网格方法由Barraquand和Latombe[8]引入,具有动态约束的机械臂时间最优运动规划的基于网格方法在[24, 40, 39]中引入。

碰撞检测的GJK算法在[50]中推导。开源碰撞检测包在开放运动规划库(OMPL)[181]和机器人操作系统(ROS)中实现。使用球体近似多面体以进行快速碰撞检测的方法在[61]中描述。

运动规划和实时避障的势场方法首先由Khatib引入,并在[73]中总结。使用势场引导搜索的基于搜索的规划器由Barraquand等人[7]描述。导航函数的构造,具有唯一局部最小值的势函数,在Koditschek和Rimon的一系列论文[78, 76, 77, 152, 153]中描述。

基于非线性优化的运动规划已在许多出版物中表述,包括Witkin和Kass[194]使用优化生成动画跳跃灯运动的经典计算机图形论文;动态非抓取操作的运动规划生成工作[103];机构最优运动的Newton算法[88];以及短突发序列动作控制的最近发展,它同时解决运动规划和反馈控制问题[3, 187]。通过细分和重连对移动机器人路径进行路径平滑由Laumond等人[82]描述。

\section{练习}

\textbf{练习10.1} 如果一条路径可以连续变形为另一条而不移动端点,则一条路径与另一条路径同伦。换句话说,它可以像橡皮筋一样拉伸和拉动,但不能切割和粘贴在一起。对于图10.2中的C空间,绘制从起点到目标的一条路径,该路径与显示的路径不同伦。

\textbf{练习10.2} 标记图10.2中的连通分量。对于每个连通分量,绘制该连通分量中一个配置的机器人图片。

\textbf{练习10.3} 假设图10.2机器人的$\theta_2$关节角度在范围$[175^\circ, 185^\circ]$内导致自碰撞。在现有C障碍物上绘制新的关节限位C障碍物,并标记$C_{\text{free}}$的所得连通分量。对于每个连通分量,绘制该连通分量中一个配置的机器人图片。

\textbf{练习10.4} 绘制对应于图10.23中障碍物和平移平面机器人的C障碍物。

\textbf{练习10.5} 编写一个程序,接受多边形机器人(相对于机器人上的参考点)的坐标和多边形障碍物的坐标作为输入,并产生相应C空间障碍物的绘图作为输出。在Mathematica中,您可能会发现函数ConvexHull有用。在MATLAB中,尝试convhull。

\textbf{练习10.6} 计算平方根可能在计算上很昂贵。对于表示为球体集合的机器人和障碍物(第10.2.2节),提供一种计算机器人和障碍物之间距离的方法,该方法最小化平方根的使用。

\textbf{练习10.7} 绘制图10.24中C障碍物以及$q_{\text{start}}$和$q_{\text{goal}}$的可见性路线图。指示最短路径。

\textbf{练习10.8} 第10.3节中描述的可见性路线图的所有边都不需要。证明如果边的任一端不切向地击中障碍物(即,它在凹顶点处击中),则C障碍物两个顶点之间的边不需要包含在路线图中。换句话说,如果边通过"碰撞"障碍物结束,它将永远不会在最短路径中使用。

\textbf{练习10.9} 实现平面中具有障碍物的点机器人的$A^*$路径规划器。平面区域是$100 \times 100$的区域。程序生成由$N$个节点和$E$条边组成的图,其中$N$和$E$由用户选择。在生成$N$个随机选择的节点后,程序应该通过边连接随机选择的节点,直到生成$E$条唯一边。与每条边相关的成本是节点之间的欧几里得距离。最后,程序应该显示图,使用$A^*$搜索图以找到节点1和$N$之间的最短路径,并显示最短路径或指示FAILURE(如果不存在路径)。启发式成本到目标是到目标节点的欧几里得距离。

\textbf{练习10.10} 修改练习10.9中的$A^*$规划器,使用等于到目标节点距离十倍的启发式成本到目标。当它们在相同图上运行时,将运行时间与原始$A^*$进行比较。(您可能需要使用大图来注意到任何效果。)使用新启发式找到的解是否最优?

\textbf{练习10.11} 修改练习10.9中的$A^*$算法以使用Dijkstra算法代替。评论当每个在相同图上运行时$A^*$和Dijkstra算法的相对运行时间。

\textbf{练习10.12} 编写一个程序,接受来自用户的多边形障碍物的顶点,以及2R机械臂的规范,根在$(x, y) = (0, 0)$,具有连杆长度$L_1$和$L_2$。每个连杆只是一条线段。通过对两个关节角度以$k$度间隔(例如,$k = 5$)采样并检查线段和多边形之间的交点,生成机器人的C空间障碍物。在工作空间中绘制障碍物,在C空间网格中,在每个与障碍物碰撞的配置处使用黑色正方形或点。(提示:此程序的核心是查看两条线段是否相交的子程序。如果段的相应无限线相交,您可以检查此交点是否在线段内。)

\textbf{练习10.13} 编写具有障碍物的2R机器人的$A^*$网格路径规划器,并在C空间上显示您找到的路径。(参见练习10.12和图10.10。)

\textbf{练习10.14} 实现轮式移动机器人的基于网格路径规划器(算法10.2),给定控制离散化。选择一种简单的方法来表示障碍物并检查碰撞。您的程序应该绘制障碍物并显示从起点到目标找到的路径。

\textbf{练习10.15} 编写平面中移动的点机器人的RRT规划器,具有障碍物。自由空间和障碍物由二维数组表示,其中每个元素对应于二维空间中的网格单元。数组中元素出现1意味着那里有障碍物,0表示该单元在自由空间中。您的程序应该绘制障碍物、形成的树,并显示从起点到目标找到的路径。

\textbf{练习10.16} 与上一个练习相同,除了障碍物现在由线段表示。线段可以被认为是障碍物的边界。

\textbf{练习10.17} 编写PRM规划器来解决与练习10.15相同的问题。

\textbf{练习10.18} 编写一个程序来实现具有点障碍物环境中2R机器人的虚拟势场。机器人的两个连杆是线段,用户指定机器人的目标配置、机器人的起始配置以及工作空间中点障碍物的位置。在机器人的每个连杆上放置两个控制点,并将工作空间势能力转换为配置空间势能力。在一个工作空间图中,绘制一个示例环境,包括几个点障碍物以及机器人在其起始和目标配置。在第二个C空间图中,在$(\theta_1, \theta_2)$上绘制势函数作为等高线图,并叠加从起始配置到目标配置的规划路径。机器人使用运动学控制律$\dot{q} = F(q)$。

看看您是否可以创建一个规划问题,该问题导致某些初始臂配置收敛到不期望的局部最小值,但对于其他初始臂配置成功找到到目标的路径。

\end{document}
